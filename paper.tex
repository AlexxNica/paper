\documentclass[letterpaper,11pt]{article}

\usepackage{graphicx}
%\usepackage{fullpage}
\usepackage{pdfpages}
\usepackage{color}
\usepackage[colorlinks=true,urlcolor=blue,citecolor=black]{hyperref}
\usepackage{url}
\usepackage[font=footnotesize,labelfont=bf]{caption}
%full name for appendix
\usepackage[title]{appendix}
\usepackage{float}
%\usepackage{parskip}
%for code
\usepackage{listings}
%for math
\usepackage{amsmath}

\linespread{1.1}
\setlength{\parindent}{30pt}
\setlength{\emergencystretch}{3em}


%opening
\title{\LARGE The Bitcoin Lightning Network:\\
	\Large Scalable Off-Chain Instant Payments}
\author{
		Joseph Poon\\
		\small\href{mailto:joseph@lightning.network}
			{\nolinkurl{joseph@lightning.network}}
	\and 
		Thaddeus Dryja\\
		\small\href{mailto:rx@awsomnet.org}
			{\nolinkurl{rx@awsomnet.org}}
	}
\date{\today\\\small DRAFT Version 0.5.9}
\begin{document}

\maketitle

\begin{abstract}

The bitcoin protocol can encompass the global financial transaction volume in
all electronic payment systems today, without a single custodial third party
holding funds or requiring participants to have any more than a computer on a
home broadband connection. A decentralized system is proposed whereby
transactions are sent over a network of micropayment channels (a.k.a. payment
channels or transaction channels) whose transfer of value occurs
off-blockchain. If Bitcoin transactions can be signed with a new sighash type
that addresses malleability, these transfers may occur between untrusted
parties along the transfer route by contracts which, in the event of
uncooperative or hostile participants, are enforceable via broadcast over the
bitcoin blockchain in the event of uncooperative or hostile participants,
through a series of decrementing timelocks.

\end{abstract}

\section{The Bitcoin Blockchain Scalability Problem}

The Bitcoin\cite{nakamoto} blockchain holds great promise for distributed
ledgers, but the blockchain as a payment platform, by itself, cannot cover the
world's commerce anytime in the near future. The blockchain is a gossip
protocol whereby all state modifications to the ledger are broadcast to all
participants. It is through this ``gossip protocol'' that consensus of the
state, everyone's balances, is agreed upon. If each node in the bitcoin network
must know about every single transaction that occurs globally, that may create
a significant drag on the ability for the network to encompass all global
financial transactions. It would instead be desirable to encompass all
transactions in a way that doesn't sacrifice the decentralization and security
that the network provides.

The payment network Visa has achieved 47,000 peak transactions per second on its
network during the 2013 holidays\cite{visa}, and averages hundreds of millions
per day. Currently, Bitcoin supports less than 7 transactions per second with a
1 megabyte block limit. If we use an average of 300 bytes per bitcoin
transaction and assumed unlimited block sizes, an equivalent capacity to peak
Visa transaction volume of 47,000/tps would be nearly 8 gigabytes per Bitcoin
block, every ten minutes on average. Continuously, that would be over 400
terabytes per year. Clearly, that isn't feasible today. No home computer in the
world can operate with that kind of bandwidth and storage. If Bitcoin is to
replace all electronic payments in the future, not just Visa, it would result in
outright collapse of the Bitcoin network, or at best, extreme centralization of
Bitcoin nodes and miners to the only ones who could afford it. This
centralization would then defeat aspects of network decentralization that make
Bitcoin secure, as the ability for entities to validate the chain is what
allows Bitcoin to ensure ledger accuracy and security. Having fewer validators
due to larger blocks not only implies fewer individuals ensuring ledger
accuracy, but also results in fewer entities that would be able to validate the
blockchain as part of the mining process, which results in encouraging miner
centralization. Extremely large blocks, for example in the above case of 8
gigabytes every 10 minutes on average, would imply that only a few parties
would be able to do block validation. This creates a great possibility that
entities will end up trusting centralized parties. Having privileged trusted
parties creates a social trap whereby the central party will not act in the
interest of an individual (principal-agent problem), e.g. rentierism by
charging higher fees to mitigate the incentive to act dishonestly. In extreme
cases, this manifests as individuals sending funds to centralized trusted
custodians who have full custody of customers funds. Such arrangements, as are
common today, create severe counterparty risk. A prerequisite to prevent that
kind of centralization from occurring would require the ability for bitcoin to
be validated by a single consumer-level computer on a home broadband
connection. By ensuring that full validation can occur cheaply, Bitcoin nodes
and miners will be able to prevent extreme centralization and trust, which
ensures extremely low transaction fees.

While it is possible that Moore's Law will continue indefinitely, and the
computational capacity for nodes to cost-effectively compute multi-gigabyte
blocks may exist in the future, it is not a certainty.

To achieve much higher than 47,000 transactions per second using Bitcoin
requires conducting transactions off the Bitcoin blockchain itself. It would be
even better if the bitcoin network supported a near-unlimited number of
transactions per second with extremely low fees for micropayments. Many
micropayments can be sent sequentially between two parties to enable any size
of payments. Micropayments would enable unbunding, less trust and
commodification of services, such as payments of per-megabyte of internet
service. To be able to achieve these micropayment use cases, however, would
require severely reducing the amount of transactions that end up being
broadcast on the global Bitcoin blockchain.

While it is possible to scale at a small level, it is absolutely not possible to
handle a large amount of micropayments on the network or to encompass all
global transactions. For bitcoin to succeed, it requires confidence that if it
were to become extremely popular, its current advantages stemming from
decentralization will continue to exist. In order for people today to believe
that Bitcoin will work tomorrow, Bitcoin needs to resolve the issue of block
size centralization effects; large blocks implicitly create trusted custodians
and significantly higher fees.

\section{A Network of Micropayment Channels Can Solve Scalability}

\begin{quote}
	``If a tree falls in the forest and no one is around to hear it, does
	it make a sound?''
\end{quote}

The above quote questions the relevance of unobserved events \textemdash if
nobody hears the tree fall, whether it made a sound or not is of no
consequence. Similarly, in the blockchain, if only two participants care about
an everyday recurring transaction, it's not necessary for all other nodes in
the bitcoin network to know about that transaction. It is instead preferable to
only have the bare minimum of information on the blockchain. By deferring
telling the entire world about every transaction, doing net settlement of their
relationship at a later date enables Bitcoin users to conduct many transactions
without bloating up the blockchain or creating trust in a centralized
counterparty. An effectively trustless structure can be achieved by using time
locks as a component to global consensus. 

Currently the solution to micropayments and scalability is to offload the
transactions to a custodian, whereby one is trusting third party custodians to
hold one's coins and to update balances with other parties. Trusting third
parties to hold all of one's funds creates counterparty risk and transaction
costs.

Instead, using a network of these micropayment channels, Bitcoin can scale to
billions of transactions per day with the computational power available on a
modern desktop computer today. Sending many payments inside a given
micropayment channel enables one to send large amounts of funds to another
party in a decentralized manner. These channels are not a separate trusted
network on top of bitcoin. They are real bitcoin transactions.

Micropayment channels\cite{wikicontracts}\cite{bitcoinjmicropay} create a
relationship between two parties to perpetually update balances, deferring what
is broadcast to the blockchain in a single transaction netting out the total
balance between those two parties. This permits the financial relationships
between two parties to be trustlessly deferred to a later date, without risk of
counterparty default. Micropayment channels use real bitcoin transactions, only
electing to defer the broadcast to the blockchain in such a way that both
parties can guarantee their current balance on the blockchain; this is not a
trusted overlay network \textemdash payments in micropayment channels are real
bitcoin communicated and exchanged off-chain.

\subsection{Micropayment Channels Do Not Require Trust}

Like the age-old question of whether the tree falling in the woods makes a
sound, if all parties agree that the tree fell at 2:45 in the afternoon, then
the tree really did fall at 2:45 in the afternoon. Similarly, if both
counterparties agree that the current balance inside a channel is 0.07 BTC to
Alice and 0.03 BTC to Bob, then that's the true balance. However, without
cryptography, an interesting problem is created: If one's counterparty
disagrees about the current balance of funds (or time the tree fell), then it
is one's word against another. Without cryptographic signatures, the blockchain
will not know who owns what. 

If the balance in the channel were 0.05 for Alice and 0.05 to Bob, and the
transaction is now 0.07 to Alice and 0.03, the network needs to know which one
is the current balance. Blockchain transactions solve this problem by using the
blockchain ledger as a timestamping system. At the same time, it is desirable
to create a system which does not actively use this timestamping system unless
absolutely necessary, as it can become costly to the network.

Instead, both parties can commit to signing a transaction and not broadcasting
this transaction. So if Alice and Bob commit funds into a 2-of-2 multisignature
address (where it requires consent from both parties to create spends), they
can agree on the current balance state. Alice and Bob can agree to create a
refund from that 2-of-2 transaction to themselves, 0.05 BTC to each. This
refund is \textit{not} broadcast on the blockchain. Either party may do so, but
they may elect to instead hold onto that transaction, knowing that they are
able to redeem funds whenever they feel comfortable doing so. By deferring
broadcast of this transaction, they may elect to change this balance at a
future date.

To update the balance, both parties create a new spend from the 2-of-2
multisignature address, for example 0.07 to Alice and 0.03 to Bob. Without
proper design, though, there is the timestamping problem of not knowing which
spend is correct: the new spend or the original refund.

The restriction on timestamping and dates, however, is not as complex as full
ordering of all transactions as in the bitcoin blockchain. In the case of
micropayment channels, only two states are required: the current correct
balance, and any old deprecated balances. There would only be a single correct
current balance, and possibly many old balances which are deprecated.

Therefore, it is possible in bitcoin to devise a bitcoin script whereby all old
transactions are invalidated, and only the new transaction is valid.
Invalidation is enforced by a bitcoin output script and dependent transactions
which force the other party to give all their funds to the channel
counterparty. By taking all funds as a penalty to give to the other, all old
transactions are thereby invalidated.

This invalidation process can exist through a process of channel consensus
where if both parties agree on current ledger states (and building new states),
then the real balance gets updated. The balance is reflected on the blockchain
only when a single party disagrees. Conceptually, this system is not an
independent overlay network; it is more a deferral of state on the current
system, as the enforcement is still occurring on the blockchain itself (albeit
deferred to future dates and transactions).

\subsection{A Network of Channels}

Thus, micropayment channels only create a relationship between two parties.
Requiring everyone to create channels with everyone else does not solve the
scalability problem. Bitcoin scalability can be achieved using a large network
of micropayment channels.

If we presume a large network of channels on the Bitcoin blockchain, and all
Bitcoin users are participating on this graph by having at least one channel
open on the Bitcoin blockchain, it is possible to create a near-infinite amount
of transactions inside this network. The only transactions that are broadcasted
on the Bitcoin blockchain prematurely are with uncooperative channel
counterparties.

%TODO: define hashlock and timelock
By encumbering the Bitcoin transaction outputs with a hashlock and timelock,
the channel counterparty will be unable to outright steal funds and Bitcoins
can be exchanged without outright counterparty theft. Further, by using
staggered timeouts, it's possible to send funds via multiple intermediaries in a
network without the risk of intermediary theft of funds.

\section{Bidirectional Payment Channels}

Micropayment channels permit a simple deferral of a transaction state to be
broadcast at a later time. The contracts are enforced by creating a
responsibility for one party to broadcast transactions before or after certain
dates. If the blockchain is a decentralized timestamping system, it is possible
to use clocks as a component of decentralized consensus\cite{lamportpaxos} to
determine data validity, as well as present states as a method to order
events\cite{lamportclocks}.

By creating timeframes where certain states can be broadcast and later
invalidated, it is possible to create complex contracts using bitcoin
transaction scripts. There has been prior work for Hub-and-Spoke Micropayment
Channels\cite{akselrod}\cite{akselrod2}\cite{todd} (and trusted payment channel
networks\cite{amikopay}\cite{impulse}) looking at building a hub-and-spoke
network today. However, Lightning Network's bidirectional micropayment channel
requires the malleability soft-fork described in Appendix A to enable
near-infinite scalability while mitigating risks of intermediate node default.

By chaining together multiple micropayment channels, it is possible to create a
network of transaction paths. Paths can be routing using a BGP-like system, and
the sender may designate a particular path to the recipient. The output scripts
are encumbered by a hash, which is generated by the recipient. By disclosing
the input to that hash, the recipient's counterparty will be able to pull funds
along the route.

\subsection{The Problem of Blame in Channel Creation}

In order to participate in this payment network, one must create a micropayment
channel with another participant on this network. 

\subsubsection{Creating an Unsigned Funding Transaction}

An initial channel Funding Transaction is created whereby one or both channel
counterparties fund the inputs of this transaction. Both parties create the
inputs and outputs for this transaction but do not sign the transaction.

The output for this Funding Transaction is a single 2-of-2 multisignature script
with both participants in this channel, henceforth named Alice and Bob. Both
participants do not exchange signatures for the Funding Transaction until they
have created spends from this 2-of-2 output refunding the original amount back
to its respective funders. The purpose of not signing the transaction allows for
one to spend from a transaction which does not yet exist. If Alice and Bob
exchange the signatures from the Funding Transaction without being able to
broadcast spends from the Funding Transaction, the funds may be locked up
forever if Alice and Bob do not cooperate (or other coin loss may occur through
hostage scenarios whereby one pays for the cooperation from the counterparty).

Alice and Bob both exchange inputs to fund the Funding Transaction (to know
which inputs are used to determine the total value of the channel), and exchange
one key to use to sign with later. This key is used for the 2-of-2 output for
the Funding Transaction; both signatures are needed to spend from the Funding
Transaction, in other words, both Alice and Bob need to agree to spend from the
Funding Transaction.

\subsubsection{Spending from an Unsigned Transaction}

The Lightning Network uses a SIGHASH\_NOINPUT transaction to spend from this
2-of-2 Funding Transaction output, as it is necessary to spend from a
transaction for which the signatures are not yet exchanged. SIGHASH\_NOINPUT,
implemented using a soft-fork, ensures transactions can be spent from before it
is signed by all parties, as transactions would need to be signed to get a
transaction ID without new sighash flags. Without SIGHASH\_NOINPUT, Bitcoin
transactions cannot be spent from before they may be broadcast \textemdash it's
as if one could not draft a contract without paying the other party first.
SIGHASH\_NOINPUT resolves this problem. See Appendix A for more information and
implementation.

Without SIGHASH\_NOINPUT, it is not possible to generate a spend from a
transaction without exchanging signatures, since spending the Funding
Transaction requires a transaction ID as part of the signature in the child's
input. A component of the Transaction ID is the parent's (Funding Transaction's)
signature, so both parties need to exchange their signatures of the parent
transaction before the child can be spent. Since one or both parties must know
the parent's signatures to spend from it, that means one or both parties are
able to broadcast the parent (Funding Transaction) before the child even exists.
SIGHASH\_NOINPUT gets around this by permitting the child to spend without
signing the input. With SIGHASH\_NOINPUT, the order of operations are to:
\begin{enumerate}
	\item Create the parent (Funding Transaction)
	\item Create the children (Commitment Transactions and all spends from
		the commitment transactions)
	\item Sign the children 
	\item Exchange the signatures for the children
	\item Sign the parent
	\item Exchange the signatures for the parent
	\item Broadcast the parent on the blockchain
\end{enumerate}

One is not able to broadcast the parent (Step 7) until Step 6 is complete. Both
parties have not given their signature to spend from the Funding Transaction
until step 6. Further, if one party fails during Step 6, the parent can
either be spent to become the parent transaction or the inputs to the parent
transaction can be double-spent (so that this entire transaction path is
invalidated).

\subsubsection{Commitment Transactions: Unenforcible Construction}

After the unsigned (and unbroadcasted) Funding Transaction has been created,
both parties sign and exchange an initial Commitment Transaction. These
Commitment Transactions spends from the 2-of-2 output of the Funding
Transaction (parent). However, only the Funding Transaction is broadcast on the
blockchain.

Since the Funding Transaction has already entered into the blockchain, and the
output are a 2-of-2 multisignature transaction which requires the agreement of
both parties to spend from, Commitment Transactions are used to express the
present balance. If only one 2-of-2 signed Commitment Transaction is exchanged
between both parties, then both parties will be sure that they are able to get
their money back after the Funding Transaction enters the blockchain. Both
parties do not broadcast the Commitment Transactions onto the blockchain until
they want to close out the current balance in the channel. They do so by
broadcasting the present Commitment Transaction.

Commitment Transactions pay out the respective current balances to each party.
A naive (broken) implementation would construct an unbroadcasted transaction
whereby there is a 2-of-2 spend from a single transaction which have two
outputs that return all current balances to both channel counterparties. This
will return all funds to the original party when creating an initial Commitment
Transaction.

%Diagram - Broken funding transaction
\begin{figure}[H]
	\makebox[\linewidth]{
		\includegraphics[width=1\linewidth]{figures/funding-broken1.pdf}
	}
	\caption{A naive broken funding transaction is described in this
		diagram. The Funding Transaction (F), designated in green, is
		broadcast on the blockchain after all other transactions are
		signed. All other transactions spending from the funding
		transactions are not yet broadcast, in case the counterparties
		wish to update their balance. Only the Funding Transaction is
		broadcast on the blockchain at this time.
	}
\end{figure}

For instance, if Alice and Bob agree to create a Funding Transaction with a
single 2-of-2 output worth 1.0 BTC (with 0.5 BTC contribution from each), they
create a Commitment Transaction where there are two 0.5 BTC outputs for Alice
and Bob. The Commitment Transactions are signed first and keys are exchanged so
either is able to broadcast the Commitment Transaction at any time contingent
upon the Funding Transaction entering into the blockchain. At this point, the
Funding Transaction signatures can safely be exchanged, as either party is able
to redeem their funds by broadcasting the Commitment Transaction.

This construction breaks, however, when one wishes to update the present
balance. In order to update the balance, they must update their Commitment
Transaction output values (the Funding Transaction has already entered into the
blockchain and cannot be changed).

When both parties agree to a new Commitment Transaction and exchange signatures
for the new Commitment Transaction, either Commitment Transactions can be
broadcast. As the output from the Funding Transaction can only be redeemed once,
only one of those transactions will be valid. For instance, if Alice and Bob
agree that the balance of the channel is now 0.4 to Alice and 0.6 to Bob, and
a new Commitment Transaction is created to reflect that, either Commitment
Transaction can be broadcast. In effect, one would be unable to restrict which
Commitment Transaction is broadcast, since both parties have signed and
exchanged the signatures for either balance to be broadcast.

%Diagram - Broken funding transaction
\begin{figure}[H]
	\makebox[\linewidth]{
		\includegraphics[width=1\linewidth]{figures/funding-broken2.pdf}
	}
	\caption{Either of the Commitment Transactions can be broadcast any any
		time by either party, only one will successfully spend from the
		single Funding Transaction. This cannot work because one party
		will not want to broadcast the most recent transaction.
	}
\end{figure}

Since either party may broadcast the Commitment Transaction at any time, the
result would be after the new Commitment Transaction is generated, the one who
receives less funds has significant incentive to broadcast the transaction which
has greater values for themselves in the Commitment Transaction outputs. As a
result, the channel would be immediately closed and funds stolen. Therefore,
one cannot create payment channels under this model.

\subsubsection{Commitment Transactions: Ascribing Blame}

Since any signed Commitment Transaction may be broadcast on the blockchain, and
only one can be successfully broadcast, it is necessary to enforce that old
Commitment Transactions do not get broadcast. It is not possible to revoke tens
of thousands of transactions in Bitcoin, so an alternate method is necessary.
Instead of active revocation enforced by the blockchain, it's necessary to
construct the channel itself in similar manner to a Fidelity Bond, whereby both
parties make commitments, and violations of these commitments are enforced by
penalties. If one party violates their agreement, then they will lose all the
money in the channel.

For this payment channel, the contract terms are that both parties commit to
broadcasting only the most recent transaction. Any broadcast of older
transactions will cause a violation of the contract, and all funds are given to
the other party as a penalty.

This can only be enforced if one is able to ascribe blame for broadcasting an
old transaction. In order to do so, one must be able to uniquely identify who
broadcast an older transaction. This can be done if each counterparty has a
uniquely identifiable Commitment Transaction. Both parties must sign the inputs
to the Commitment Transaction which the other party is responsible for
broadcasting. Since one has a version of the Commitment Transaction that is
signed by the other party, one can only broadcast one's own version of the
Commitment Transaction.

For the Lightning Network, all spends from the Funding Transaction output,
Commitment Transactions, have two half-signed transactions. One Commitment
Transaction in which Alice signs and gives to Bob (C1b), and another which Bob
signs and gives to Alice (C1a). These two Commitment Transactions spend from the
same output (Funding Transaction), and have different contents; only one can be
broadcast on the blockchain, as both pairs of Commitment Transactions spend from
the same Funding Transaction. Either party may broadcast their received
Commitment Transaction by signing their version and including the counterparty's
signature. For example, Bob can broadcast Commitment C1b, since he has already
received the signature for C1b from Alice \textemdash he includes Alice's
signature and signs C1b himself. The transaction will be a valid spend from the
Funding Transaction's 2-of-2 output requiring both Alice and Bob's signature.

%Diagram - Broken funding transaction
\begin{figure}[H]
	\makebox[\linewidth]{
		\includegraphics[width=1\linewidth]{figures/funding-broken3.pdf}
	}
	\caption{Purple boxes are unbroadcasted transactions which only Alice
		can broadcast. Blue boxes are unbroadcasted transaction which
		only Bob can broadcast. Alice can only broadcast Commitment 1a,
		Bob can only broadcast Commitment 1b. Only one Commitment
		Transaction can be spent from the Funding Transaction output.
		Blame is ascribed, but either one can still be spent with no
		penalty.
	}
\end{figure}

However, even with this construction, one has only merely allocated blame. It is
not yet possible to enforce this contract on the Bitcoin blockchain. Bob still
trusts Alice not to broadcast an old Commitment Transaction. At this time, he is
only able to prove that Alice has done so via a half-signed transaction proof.

\subsection{Creating a Channel with Contract Revocation}

To be able to actually enforce the terms of the contract, it's necessary to
construct a Commitment Transaction (along with its spends) where one is able
to revoke a transaction. This revocation is achievable by using data about when
a transaction enters into a blockchain and using the maturity of the transaction
to determine validation paths.

\subsection{Sequence Number Maturity}

Mark Freidenbach has proposed that Sequence Numbers can be enforcible via a
relative block maturity of the parent transaction via a
soft-fork\cite{seqnum}. This would allow some basic ability to ensure some form
of relative block confirmation time lock on the spending script. In addition, an
additional opcode, OP\_CHECKSEQUENCEVERIFY (a.k.a.
OP\_RELATIVECHECKLOCKTIMEVERIFY), would permit further abilities,
including allowing a stop-gap solution before a more permanent solution for
resolving transaction malleability. A future version of this paper will include
proposed solutions.

To summarize, Bitcoin was released with a sequence number which was only
enforced in the mempool of unconfirmed transactions. The original behavior
permitted transaction replacement by replacing transactions in the mempool with
newer transactions if they have a higher sequence number. Due to transaction
replacement rules, it is not enforced due to denial of service attack risks. It
appears as though the intended purpose of the sequence number is to replace
unbroadcasted transactions. However, this higher sequence number replacement
behavior is unenforcible. One cannot be assured that old versions of
transactions were replaced in the mempool and a block contains the most recent
version of the transaction. A way to enforce transaction versions off-chain is
via time commitments.

A Revocable Transaction spends from a unique output where the transaction has a
unique type of output script. This parent's output has two redemption paths
where the first can be redeemed immediately, and the second can only be redeemed
if the child has a minimum number of confirmations between transactions. This is
achieved by making the sequence number of the child transaction require a
minimum number of confirmations from the parent. In essence, this new sequence
number behavior will only permit a spend from this output to be valid if the
number of blocks between the output and the redeeming transaction is above a
specified block height.

A transaction can be revoked with this sequence number behavior by creating a
restriction with some defined number of blocks defined in the sequence number,
which will result in the spend being only valid after the parent has entered
into the blockchain for some defined number of blocks. This creates a structure
whereby the parent transaction with this output becomes a bonded deposit,
attesting that there is no revocation. A time period exists which anyone on the
blockchain can refute this attestation by broadcasting a spend immediately
after the transaction is broadcast.

If one wishes to permit revocable transactions with a 1000-confirmation delay,
the output transaction construction would remain a 2-of-2 multisig:

\begin{lstlisting}
2 <Alice1> <Bob1> 2 OP_CHECKMULTISIG
\end{lstlisting}

However, the child spending transaction would contain a nSequence value of 1000.
Since this transaction requires the signature of both counterparties to be
valid, both parties include the nSequence number of 1000 as part of the
signature. Both parties may, at their discretion, agree to create another
transaction which supersedes that transaction without any nSequence number.

This construction, a Revocable Sequence Maturity Contract (RSMC), creates
two paths, with very specific contract terms. 

The contract terms are:
\begin{enumerate}
	\item All parties pay into a contract with an output enforcing this
		contract
	\item Both parties may agree to send funds to some contract, with some
		waiting period (1000 confirmations in our example script). This
		is the revocable output balance.
	\item One or both parties may elect to not broadcast (enforce) the
		payouts until some future date; either party may redeem the
		funds after the waiting period at any time.
	\item If neither party has broadcast this transaction (redeemed the
		funds), they may revoke the above payouts if and only if both
		parties agree to do so by placing in a new payout term in a
		superseding transaction payout. The new transaction payout can
		be immediately redeemed after the contract is disclosed to the
		world (broadcast on the blockchain).
	\item In the event that the contract is disclosed and the new payout
		structure is not redeemed, the prior revoked payout terms may be
		redeemed by either party (so it is the responsibility of either
		party to enforce the new terms).
\end{enumerate}

The pre-signed child transaction can be redeemed after the parent transaction
has entered into the blockchain with 1000 confirmations, due to the child's
nSequence number on the input spending the parent.

In order to revoke this signed child transaction, both parties just agree to
create another child transaction with the default field of the nSequence number
of MAX\_INT, which has special behavior permitting spending at any time.

This new signed spend supersedes the revocable spend so long as the new signed
spend enters into the blockchain within 1000 confirmations of the parent
transaction entering into the blockchain. In effect, if Alice and Bob agree to
monitor the blockchain for incorrect broadcast of Commitment Transactions, the
moment the transaction gets broadcast, they are able to spend using the
superseding transaction immediately. In order to broadcast the revocable spend
(deprecated transaction), which spends from the same output as the superseding
transaction, they must wait 1000 confirmations. So long as both parties watch
the blockchain, the revocable spend will never enter into the transaction if
either party prefers the superseding transaction.

Using this construction, anyone could create a transaction, not broadcast the
transaction, and then later create incentives to not ever broadcast that
transaction in the future via penalties. This permits participants on the
Bitcoin network to defer many transactions from ever hitting the blockchain.

\subsubsection{Timestop}

To mitigate a flood of transactions by a malicious attacker requires a credible
threat that the attack will fail. 

Greg Maxwell proposed using a timestop to mitigate a malicious flood on the
blockchain:

\begin{quote}
	There are many ways to address this [flood risk] which haven't been
	adequately explored yet \textemdash for example, the clock can stop
	when blocks are full; turning the security risk into more hold-up delay
	in the event of a dos attack.\cite{gregtimestop}
\end{quote}

This can be mitigated by allowing the miner to specify whether the current
(fee paid) mempool is presently being flooded with transactions. They can enter
a ``1'' value into the last bit in the version number of the block header. If
the last bit in the block header contains a ``1'', then that block will not
count towards the relative height maturity for the nSequence value and the block
is designated as a congested block. There is an uncongested block height (which
is always lower than the normal block height). This block height is used for the
nSequence value, which only counts block maturity (confirmations).

A miner can elect to define the block as a congested block or not. The default
code could automatically set the congested block flag as ``1'' if the mempool is
above some size and the average fee for that set size is above some value.
However, a miner has full discretion to change the rules on what automatically
sets as a congested block, or can select to permanently set the congestion flag
to be permanently on or off. It's expected that most honest miners would use the
default behavior defined in their miner and not organize a 51\% attack.

For example, if a parent transaction output is spent by a child with a nSequence
value of 10, one must wait 10 confirmations before the transaction becomes
valid. However, if the timestop flag has been set, the counting of confirmations
stops, even with new blocks. If 6 confirmations have elapsed (4 more are
necessary for the transaction to be valid), and the timestop block has been set
on the 7th block, that block does not count towards the nSequence requirement of
10 confirmations; the child is still at 6 blocks for the relative confirmation
value. Functionally, this will be stored as some kind of auxiliary timestop
block height which is used only for tracking the timestop value. When the
timestop bit is set, all transactions using an nSequence value will stop
counting until the timestop bit has been unset. This gives sufficient time and
block-space for transactions at the current auxiliary timestop block height to
enter into the blockchain, which can prevent systemic attackers from
successfully attacking the system.

However, this requires some kind of flag in the block to designate whether it
is a timestop block. For full SPV compatibility (Simple Payment Verification;
lightweight clients), it is desirable for this to be within the 80-byte block
header instead of in the coinbase. There are two places which may be a good
place to put in this flag in the block header: in the block time and in the
block version. The block time may not be safe due to the last bits being used as
an entropy source for some ASIC miners, therefore a bit may need to be consumed
for timestop flags. Another option would be to hardcode timestop activation as a
hard consensus rules (e.g. via block size), however this may make things less
flexible. By setting sane defaults for timestop rules, these rules can be
changed without consensus soft-forks.

If the block version is used as a flag, the contextual information must match
the Chain ID used in some merge-mined coins.

\subsubsection{Revocable Commitment Transactions}

By combining the ascribing of blame as well as the revocable transaction, one
is able to determine when a party is not abiding by the terms of the contract,
and enforce penalties without trusting the counterparty.

%Diagram - Funding Transaction and all Commitment Spends 1
\begin{figure}[H]
	\makebox[\linewidth]{
		\includegraphics[width=1\linewidth]{figures/funding-full.pdf}
	}
	\caption{The Funding Transaction F, designated in green, is broadcast
		on the blockchain after all other transactions are signed. All
		transactions which only Alice can broadcast are in purple. All
		transactions which only Bob can broadcast is are blue. Only the
		Funding Transaction is broadcast on the blockchain at this
		time.
	}
\end{figure}

The intent of creating a new Commitment Transaction is to invalidate all old
Commitment Transactions when updating the new balance with a new Commitment
Transaction. Invalidation of old transactions can happen by making an output be
a Revocable Sequence Verification Contract (RSMC). To invalidate a transaction,
a superseding transaction will be signed and exchanged by both parties that 
gives all funds to the counterparty in the event an older transaction is
incorrectly broadcast. The incorrect broadcast is identified by creating two
different Commitment Transactions with the same final balance outputs, however
the payment to oneself is encumbered by a RSMC.

In effect, there are two Commitment Transactions from a single Funding
Transaction 2-of-2 outputs. Of these two Commitment Transactions, only one can
enter into the blockchain. Each party within a channel has one version of this
contract. So if this is the first Commitment Transaction pair, Alice's
Commitment Transaction is defined a C1a, and Bob's Commitment Transaction is
defined as C1b. By broadcasting a Commitment Transaction, one is requesting for
the channel to close out and end. The first two outputs for the Commitment
Transaction include a Delivery Transaction (payout) of the present unallocated
balance to the channel counterparties. If Alice broadcasts C1a, one of the
output is spendable by D1a, which sends funds to Bob. For Bob, C1b is spendable
by D1b, which sends funds to Alice. The Delivery Transaction (D1a/D1b) is
immediately redeemable and is not encumbered in any way in the event the
Commitment Transaction is broadcast.

For each party's Commitment Transaction, they are attesting that they are
broadcasting the most recent Commitment Transaction which they own. Since they
are attesting that this is the current balance, the balance paid to the
counterparty is assumed to be true, since one has no direct benefit by paying
some funds to the counterparty as a penalty.

The balance paid to the person who broadcast the Commitment Transaction,
however, is unverified. The participants on the blockchain have no idea if the
Commitment Transaction is the most recent or not. If they do not broadcast their
most recent version, they will be penalized by taking all the funds in the
channel and giving it to the counterparty. Since their own funds are encumbered
in their own RSMC, they will only be able to claim their funds after some set
number of confirmations after the Commitment Transaction has been included in a
block (in our example, 1000 confirmations). If they do broadcast their most
recent Commitment Transaction, there should be no revocation transaction
superseding the revocable transaction, so they will be able to receive their
funds after some set amount of time (1000 confirmations).

By knowing who broadcast the Commitment Transaction and encumbering one's own
payouts to be locked up for a predefined period of time, both parties will be
able to revoke the Commitment Transaction in the future.

\subsubsection{Redeeming Funds from the Channel: Cooperative Counterparties}

Either party may redeem the funds from the channel. However, the party that
broadcasts the Commitment Transaction must wait for the predefined number of
confirmations described in the RSMC. The counterparty which did not broadcast
the Commitment Transaction may redeem the funds immediately.

For example, if the Funding Transaction is committed with 1 BTC (half to each
counterparty) and Bob broadcasts the most recent Commitment Transaction, C1b, he
must wait 1000 confirmations to receive his 0.5 BTC, while Alice can spend 0.5
BTC. For Alice, this transaction is fully closed if Alice agrees that Bob
broadcast the correct Commitment Transaction (C1b).

\begin{figure}[H]
	\makebox[\linewidth]{
		\includegraphics[width=1\linewidth]{figures/funding-full-bob-spend.pdf}
	}
	\caption{When Bob broadcasts C1b, Alice can immediately redeem her
		portion. Bob must wait 1000 confirmations. When the block is
		immediately broadcast, it is in this state. Transactions in
		green are transactions which are committed into the blockchain.
	}
\end{figure}

After the Commitment Transaction has been in the blockchain for 1000 blocks,
Bob can then broadcast the Revocable Delivery transaction. He must wait 1000
blocks to prove he has not revoked this Commitment Transaction (C1b). After
1000 blocks, the Revoable Delivery transaction will be able to be included in a
block. If it is attempted to be included in a block before 1000 confirmations,
the transaction will be invalid up until after 1000 confirmations have passed
(at which point it will become valid if the output has not yet been redeemed).

\begin{figure}[H]
	\makebox[\linewidth]{
		\includegraphics[width=1\linewidth]{figures/funding-full-bob-redeem.pdf}
	}
	\caption{Alice agrees that Bob broadcast the correct Commitment
		Transaction and 1000 confirmations have passed. Bob then is able
		to broadcast the Revocable Delivery (RD1b) transaction on the
		blockchain.
	}
\end{figure}

After Bob broadcasts the Revocable Delivery transaction, the channel is fully
closed for both Alice and Bob, everyone has received the funds which they both
agree are the current balance they each own in the channel.

If it was instead Alice who broadcast the Commitment Transaction (C1a), she is
the one who must wait 1000 confirmations instead of Bob.

\subsubsection{Creating a new Commitment Transaction and Revoking Prior
Commitments}

While each party may close out the most recent Commitment Transaction at any
time, they may also elect to create a new Commitment Transaction and invalidate
the old one.

Suppose Alice and Bob now want to update their current balances from 0.5 BTC
each refunded to 0.6 BTC for Bob and 0.4 BTC for Alice. When they both agree to
do so, they generate a new pair of Commitment Transactions.

\begin{figure}[H]
	\makebox[\linewidth]{
		\includegraphics[width=1\linewidth]{figures/newcommit-simple.pdf}
	}
	\caption{Four possible transactions can exist, a pair with the old
		commitments, and another pair with the new commitments. Each
		party inside the channel can only broadcast half of the total
		commitments (two each). There is no explicit enforcement
		preventing any particular Commitment being broadcast other than
		penalty spends, as they are all valid unbroadcasted spends.
		The Revocable Commitment still exists with the C1a/C1b pair, but
		are not displayed for brevity.
	}
\end{figure}

When a new pair of Commitment Transactions (C2a/C2b) is agreed upon, both
parties will sign and exchange signatures for the new Commitment Transaction,
then invalidate the old Commitment Transaction. This invalidation occurs by
having both parties sign a Breach Remedy Transaction (BR1), which supersedes the
Revocable Delivery Transaction (RD1). Each party hands to the other a
half-signed revocation (BR1) from their own Revocable Delivery (RD1), which is a
spend from the Commitment Transaction. The Breach Remedy Transaction will send
all coins to the counterparty within the current balance of the channel. For
example, if Alice and Bob both generate a new pair of Commitment Transactions
(C2a/C2b) and invalidate prior commitments (C1a/C1b), and later Bob incorrectly
broadcasts C1b on the blockchain, Alice can take all of Bob's money from the
channel. Alice can do this because Bob has proved to Alice via penalty that he
will never broadcast C1b, since the moment he broadcasts C1b, Alice is able to
take all of Bob's money in the channel. In effect, by constructing a Breach
Remedy transaction for the counterparty, one has attested that one will not be
broadcasting any prior commitments. The counterparty can accept this, because
they will get all the money in the channel when this agreement is violated.

\begin{figure}[H]
	\makebox[\linewidth]{
		\includegraphics[width=1.3\linewidth]{figures/newcommit-br.pdf}
	}
	\caption{
		When C2a and C2b exists, both exchange Breach Remedy
		transactions. Both parties now have explicit economic incentive
		to avoid broadcasting old Commitment Transactions (C1a/C1b).
		If either party wishes to close out the channel, they will only
		use C2a (Alice) or C2b (Bob). If Alice broadcasts C1a, all her
		money will go to Bob. If Bob broadcasts C1b, all her money will
		go to Alice. See previous figure for C2a/C2b outputs.
	}
\end{figure}

Due to this fact, one will likely delete all prior Commitment Transactions when
a Breach Remedy Transaction has been passed to the counterparty. If one
broadcasts an incorrect (deprecated and invalidated Commitment Transaction), all
the money will go to one's counterparty. For example, if Bob broadcasts C1b, so
long as Alice watches the blockchain within the predefined number of blocks (in
this case, 1000 blocks), Alice will be able to take all the money in this
channel by broadcasting RD1b. Even if the present balance of the Commitment
state (C2a/C2b) is 0.4 BTC to Alice and 0.6 BTC to Bob, because Bob violated the
terms of the contract, all the money goes to Alice as a penalty. Functionally,
the Revocable Transaction acts as a proof to the blockchain that Bob has
violated the terms in the channel and this is programatically adjudicated by the
blockchain.

\begin{figure}[H]
	\makebox[\linewidth]{
		\includegraphics[width=1.3\linewidth]{figures/newcommit-penalty.pdf}
	}
	\caption{
		Transactions in green are committed to the blockchain. Bob
		incorrectly broadcasts C1b (only Bob is able to broadcast
		C1b/C2b). Because both agreed that the current state is the
		C2a/C2b Commitment pair, and have attested to each party that
		old commitments are invalidated via Breach Remedy Transactions,
		Alice is able to broadcast BR1b and take all the money in the
		channel, provided she does it within 1000 blocks after C1b is
		broadcast.
	}
\end{figure}

However, if Alice does not broadcast BR1b within 1000 blocks, Bob may be able to
steal some money, since his Revocable Delivery Transaction (RD1b) becomes valid
after 1000 blocks. When an incorrect Commitment Transaction is broadcast, only
the Breach Remedy Transaction can be broadcast for 1000 blocks (or whatever
number of confirmations both parties agree to). After 1000 block confirmations,
both the Breach Remedy (BR1b) and Revocable Delivery Transactions (RD1b) are
able to be broadcast at any time. Breach Remedy transactions only have
exclusivity within this predefined time period, and any time after of that is
functionally an expiration of the statute of limitations \textemdash according
to Bitcoin blockchain consensus, the time for dispute has ended. 

For this reason, one should periodically monitor the blockchain to see if one's
counterparty has broadcast an invalidated Commitment Transaction, or delegate a
third party to do so. A third party can be delegated by only giving the Breach
Remedy transaction to this third party. They can be incentivized to watch the
blockchain broadcast such a transaction in the event of counterparty
maliciousness by giving these third parties some fee in the output. Since the
third party is only able to take action when the counterparty is acting
maliciously, this third party does not have any power to force close of the
channel.

\subsubsection{Process for Creating Revocable Commitment Transactions}

To create revocable Commitment Transactions, it requires proper construction of
the channel from the beginning, and only signing transactions which may be
broadcast at any time in the future, while ensuring that one will not lose out
by uncooperative or malicious counterparties. This requires determining which
public key to use for new commitments, as using SIGHASH\_NOINPUT requires using
unique keys for each Commitment Transaction RSMC (and HTLC) output. We use $P$
to designate pubkeys and $K$ to designate the corresponding private key used to
sign.

When generating the first Commitment Transaction, Alice and Bob agree to create
a multisig output from a Funding Transaction with a single $multisig(P_{AliceF},
P_{BobF})$ output, funded from 0.5 from Alice and Bob for a total of 1 BTC. This
output is a Pay to Script Hash\cite{p2sh} transaction, which requires both Alice
and Bob to both agree to spend from the Funding Transaction. They do not yet
make the Funding Transaction (F) spendable. Additionally, $P_{AliceF}$ and
$P_{BobF}$ are only used for the Funding Transaction, they are not used for
anything else.

Since the Delivery transaction is just a P2PKH output (bitcoin addresses
beginning with 1) or P2SH transaction (commonly recognized as addresses
beginning with the 3) which the counterparties designates
beforehand, this can be generated as an output of $P_{AliceD}$ and $P_{BobD}$.
For simplicity, these output addresses will remain the same throughout the
channel, since its funds are fully controlled by its designated recipient after
the Commitment Transaction enters the blockchain. If desired, but not
necessary, both parties may update and change $P_{AliceD}$ and $P_{BobD}$ for
future Commitment Transactions.

Both parties exchange pubkeys they intend to use for the RSMC (and HTLC
described in future sections) for the Commitment Transaction. Each set of
Commitment Transactions use their own public keys and are not ever reused. Both
parties may already know all future pubkeys by using a BIP 0032\cite{bip32} HD
Wallet construction by exchanging Master Public Keys during channel
construction. If they wish to generate a new Commitment Transaction pair
C2a/C2b, they use multisig($P_{AliceRSMC2}$, $P_{BobRSMC2}$) for the RSMC
output.

After both parties know the output values from the Commitment Transactions, both
parties create the pair of Commitment Transactions, e.g. C2a/C2b, but do not
exchange signatures for the Commitment Transactions. They both sign the
Revocable Delivery transaction (RD2a/RD2b) and exchange the signatures. Bob
signs RD1a and gives it to Alice (using $K_{BobRSMC2}$), while Alice signs RD1b
and gives it to Bob (using $K_{AliceRSMC2}$).

When both parties have the Revocable Delivery transaction, they exchange
signatures for the Commitment Transactions. Bob signs C1a using $K_{BobF}$ and
gives it to Alice, and Alice signs C1b using $K_{AliceF}$ and gives it to Bob.

At this point, the prior Commitment Transaction as well as the new Commitment
Transaction can be broadcast; both C1a/C1b and C2a/C2b are valid. (Note that
Commitments older than the prior Commitment are invalidated via penalties.) In
order to invalidate C1a and C1b, both parties exchange Breach Remedy Transaction
(BR1a/BR1b) signatures for the prior commitment C1a/C1b. Alice sends BR1a to Bob
using $K_{AliceRSMC1}$, and Bob sends BR1b to Alice using $K_{BobRSMC1}$. When
both Breach Remedy signatures have been exchanged, the channel state is now at
the current Commitment C2a/C2b and the balances are now committed.

However, instead of disclosing the BR1a/BR1b signatures, it's also possible to
just disclose the private keys to the counterparty. This is more effective as
described later in the key storage section. One can disclose the private keys
used in one's own Commitment Transaction. For example, if Bob wishes to
invalidate C1b, he sends his private keys used in C1b to Alice (he does
\textit{NOT} disclose his keys used in C1a, as that would permit coin theft).
Similarly, Alice discloses all her private key outputs in C1a to Bob to
invalidate C1a.

If Bob incorrectly broadcasts C1b, then because Alice has all the private keys
used in the outputs of C1b, she can take the money. However, only Bob is able
to broadcast C1b. To prevent this coin theft risk, Bob should destroy all old
Commitment Transactions.

\subsection{Cooperatively Closing Out a Channel}

Both parties are able to send as many payments to their counterparty as they
wish, as long as they have funds available in the channel, knowing that in the
event of disagreements they can broadcast to the blockchain the current state
at any time.

In the vast majority of cases, all the outputs from the Funding Transaction
will never be broadcast on the blockchain. They are just there in case the
other party is non-cooperative, much like how a contract is rarely enforced in
the courts. A proven ability for the contract to be enforced in a deterministic
manner is sufficient incentive for both parties to act honestly.

When either party wishes to close out a channel cooperatively, they will be
able to do so by contacting the other party and spending from the Funding
Transaction with an output of the most current Commitment Transaction directly
with no script encumbering conditions. No further payments may occur in the
channel.

\begin{figure}[H]
	\makebox[\linewidth]{
		\includegraphics[width=1\linewidth]{figures/cooperative-close.pdf}
	}
	\caption{If both counterparties are cooperative, they take the
		balances in the current Commitment Transaction and spend from
		the Funding Transaction with a Exercise Settlement Transaction
		(ES). If the most recent Commitment Transaction gets broadcast
		instead, the payout (less fees) will be the same.
	}
\end{figure}

The purpose of closing out cooperatively is to reduce the number of
transactions that occur on the blockchain and both parties will be able to
receive their funds immediately (instead of one party waiting for the
Revocation Delivery transaction to become valid).

Channels may remain in perpetuity until they decide to cooperatively close out
the transaction, or when one party does not cooperate with another and the
channel gets closed out and enforced on the blockchain.

\subsection{Bidirectional Channel Implications and Summary}

By ensuring channels can update only with the consent of both parties, it is
possible to construct channels which perpetually exist in the blockchain. Both
parties can update the balance inside the channel with whatever output balances
they wish, so long as it's equal or less than the total funds committed inside
the Funding Transaction; balances can move in both directions. If one party
becomes malicious, either party may immediately close out the channel and
broadcast the most current state to the blockchain. By using a fidelity bond
construction (Revocable Delivery Transactions), if a party violates the terms of
the channel, the funds will be sent to the counterparty, provided the proof of
violation (Breach Remedy Transaction) is entered into the blockchain in a timely
manner. If both parties are cooperative, the channel can remain open
indefinitely, possibly for many years.

This type of construction is only possible because adjudication occurs
programatically over the blockchain as part of the Bitcoin consensus, so one
does not need to trust the other party. As a result, one's channel counterparty
does not possess full custody or control of the funds.

\section{Hashed Timelock Contract (HTLC)}

A bidirectional payment channel only permits secure transfer of funds inside a
channel. To be able to construct secure transfers using a network of channels
across multiple hops to the final destination requires an additional
construction, a Hashed Timelock Contract (HTLC).

The purpose of an HTLC is to allow for global state across multiple nodes via
hashes. This global state is ensured by time commitments and time-based
unencumbering of resources via disclosure of preimages. Transactional "locking"
occurs globally via commitments, at any point in time a single participant is
responsible for disclosing to the next participant whether they have knowledge
of the preimage $R$. This construction does not require custodial trust in one's
channel counterparty, nor any other participant in the network.

In order to achieve this, an HTLC must be able to create certain transactions
which are only valid after a certain date, using nLockTime, as well as
information disclosure to one's channel counterparty. Additionally, this data
must be revocable, as one must be able to undo an HTLC.

An HTLC is also a channel contract with one's counterparty which is enforcible
via the blockchain. The counterparties in a channel agree to the following
terms for a Hashed Timelock Contract:

\begin{enumerate}
	\item If Bob can produce to Alice an unknown 20-byte random input data
		$R$ from a known hash $H$, within three days, then Alice will
		settle the contract by paying Bob 0.1 BTC.

	\item If three days have elapsed, then the above clause is null and void
		and the clearing process is invalidated, both parties must not
		attempt to settle and claim payment after three days.

	\item Either party may (and should) pay out according to the terms of
		this contract in any method of the participants choosing and
		close out this contract early so long as both participants in
		this contract agree.

	\item Violation of the above terms will incur a maximum penalty of the
		funds locked up in this contract, to be paid to the
		non-violating counterparty as a fidelity bond.

\end{enumerate}

For clarity of examples, we use days for HTLCs and block height for RSMCs. In
reality, the HTLC should also be defined as a block height (e.g. 3 days is
equivalent to 432 blocks).

In effect, one desires to construct a payment which is contingent upon knowledge
of $R$ by the recipient within a certain timeframe. After this timeframe, the
funds are refunded back to the sender.

Similar to RSMCs, these contract terms are programatically enforced on the
Bitoin blockchain and do not require trust in the counterparty to adhere to the
contract terms, as all violations are penalized via unilaterally enforced
fidelity bonds, which are constructed using penalty transactions spending from
commitment states. If Bob knows $R$ within three days, then he can redeem the
funds by broadcasting a transaction; Alice is unable to withhold the funds in
any way, because the script returns as valid when the transaction is spent on
the Bitcoin blockchain.

An HTLC is an additional output in a Commitment Transaction with a unique
output script:

\begin{lstlisting}
OP_IF 
	OP_HASH160 <Hash160(R)> OP_EQUALVERIFY
	2 <Alice2> <Bob2> OP_CHECKMULTISIG
OP_ELSE 
	2 <Alice1> <Bob1> OP_CHECKMULTISIG
OP_ENDIF
\end{lstlisting}

Conceptually, this script has two possible paths spending from a single HTLC
output. The first path (defined in the OP\_IF) sends funds to Bob if Bob can
produce $R$. The second path is redeemed using a 3-day timelocked refund to
Alice. The 3-day timelock is enforced using nLockTime from the spending
transaction.

\subsection{Non-revocable HTLC Construction}

\begin{figure}[H]
	\makebox[\linewidth]{
		\includegraphics[width=1\linewidth]{figures/htlc-concept.pdf}
	}
	\caption{This is a non-functional naive implementation of an HTLC. Only
		the HTLC path from the Commitment Transaction is displayed. Note
		that there are two possible spends from an HTLC output. If Bob
		can produce the preimage $R$ within 3 days and he can redeem
		path 1. After three days, Alice is able to broadcast path 2.
		When 3 days have elapsed either is valid. This model, however
		doesn't work with multiple Commitment Transactions.
	}
\end{figure}

If $R$ is produced within 3 days, then Bob can redeem the funds by broadcasting
the ``Delivery'' transaction. A requirement for the ``Delivery'' transaction to
be valid requires $R$ to be included with the transaction. If $R$ is not
included, then the ``Delivery'' transaction is invalid. However, if 3 days have
elapsed, the funds can be sent back to Alice by broadcasting transaction
``Timeout''. When 3 days have elapsed and $R$ has been disclosed, either
transaction may be valid.

It is within both parties individual responsibility to ensure that they can get
their transaction into the blockchain in order to ensure the balances are
correct. For Bob, in order to receive the funds, he must either broadcast the
``Delivery'' transaction on the Bitcoin blockchain, or otherwise settle with
Alice (while cancelling the HTLC). For Alice, she must broadcast the ``Timeout''
3 days from now to receive the refund, or cancel the HTLC entirely with Bob.

Yet this kind of simplistic construction has similar problems as an incorrect
bidirectional payment channel construction. When an old Commitment Transaction
gets broadcast, either party may attempt to steal funds as both paths may be
valid after the fact. For example, if $R$ gets disclosed 1 year later, and an
incorrect Commitment Transaction gets broadcast, both paths are valid and are
redeemable by either party; the contract is not yet enforcible on the
blockchain. Closing out the HTLC is absolutely necessary, because in order for
Alice to get her refund, she must terminate the contract and receive her refund.
Otherwise, when Bob discovers $R$ after 3 days have elapsed, he may be able to
steal the funds which should be going to Alice. With uncooperative
counterparties it's not possible to terminate an HTLC without broadcasting it to
the bitcoin blockchain as the uncooperative party is unwilling to create a new
Commitment Transaction. 

\subsection{Off-chain Revocable HTLC}

To be able to terminate this contract off-chain without a broadcast to the
Bitcoin blockchain requires embedding RSMCs in the output, which will have a
similar construction to the bidirectional channel.

\begin{figure}[H]
	\vspace*{-2cm}
	\makebox[\linewidth]{
		\includegraphics[width=1.3\linewidth]{figures/htlc-functional.pdf}
	}
	\caption{If Alice broadcasts C2a, then the left half will execute. If
		Bob broadcasts C2b, then the right half will execute. Either
		party may broadcast their Commitment transaction at any time.
		HTLC Timeout is only valid after 3 days. HTLC Executions can
		only be broadcast if the preimage to the hash $R$ is known.
		Prior Commitments (and their dependent transactions) are not
		displayed for brevity.
	}
\end{figure}

Presume Alice and Bob wish to update their balance in the channel at Commitment
1 with a balance of 0.5 to Alice and 0.5 to Bob.

Alice wishes to send 0.1 to Bob contingent upon knowledge of $R$ within 3 days,
after 3 days she wants her money back if Bob does not produce $R$.

The new Commitment Transaction will have a full refund of the current balance to
Alice and Bob (Outputs 0 and 1), with output 2 being the HTLC, which describes
the funds in transit. As 0.1 will be encumbered in an HTLC, Alice's balance is
reduced to 0.4 and Bob's remains the same at 0.5.

This new Commitment Transaction (C2a/C2b) will have an HTLC output with two
possible spends. Each spend is different depending on each counterparty's
version of the Commitment Transaction. Similar to the bidirectional payment
channel, when one party broadcasts their Commitment, payments to the
counterparty will be assumed to be valid and not invalidated. This can occur
because when one broadcasts a Commitment Transaction, one is attesting this is
the most recent Commitment Transaction. If it is the most recent, then one
is also attesting that the HTLC exists and was not invalidated before, so
potential payments to one's counterparty should be valid.

Note that HTLC transaction names (beginning with the letter H) will begin with
the number 1, whose values do not correlate with Commitment Transactions. This
is simply the first HTLC transaction. HTLC transactions may persist between
Commitment Transactions. Each HTLC has 4 keys per side of the
transaction (C2a and C2b) for a total of 8 keys per counterparty.

The HTLC output in the Commitment Transaction has two sets of keys per
counterparty in the output. 

For Alice's Commitment Transaction (C2a), the HTLC output script requires
$multisig(P_{Alice2}, P_{Bob2})$ encumbered by disclosure of $R$, as well as
$multisig(P_{Alice1}, P_{Bob1})$ with no encumbering.

For Bob's Commitment Transaction (C2b), the HTLC output script requires
$multisig(P_{Alice6}, P_{Bob6})$ encumbered by disclosure of $R$, as well as
$multisig(P_{Alice5}, P_{Bob5})$ with no encumbering.

The HTLC output states are different depending upon which Commitment Transaction
is broadcast.

\subsubsection{HTLC when the Sender Broadcasts the Commitment Transaction}

For the sender (Alice), the ``Delivery'' transaction is sent as an HTLC
Execution Delivery transaction (HED1a), which is not encumbered in an RSMC. It
assumes that this HTLC has never been terminated off-chain, as Alice is
attesting that the broadcasted Commitment Transaction is the most recent. If Bob
can produce the preimage $R$, he will be able to redeem funds from the HTLC
after the Commitment Transaction is broadcast on the blockchain. This
transaction consumes $multisig(P_{Alice2}, P_{Bob2})$ if Alice broadcasts her
Commitment C2a. Only Bob can broadcast HED1a since only Alice gave her signature
for HED1a to Bob.

However, if 3 days have elapsed since forming the HTLC, then Alice will be able
broadcast a ``Timeout'' transaction, the HTLC Timeout transaction (HT1a). This
transaction is an RSMC. It consumes the output $multisig(P_{Alice1}, P_{Bob1})$
without requiring disclosure of $R$ if Alice broadcasts C2a. This transaction
cannot enter into the blockchain until 3 days have elapsed. The output for this
transaction is an RSMC with $multisig(P_{Alice3}, P_{Bob3})$ with relative
maturity of 1000 blocks, and $multisig(P_{Alice4}, P_{Bob4})$ with no
requirement for confirmation maturity. Only Alice can broadcast HT1a since only
Bob gave his signature for HT1a to Alice.

After HT1a enters into the blockchain and 1000 block confirmations, a HTLC
Timeout Revocable Delivery transaction (HTRD1a) may be broadcast by Alice which
consumes $multisig(P_{Alice3}, P_{Bob3})$. Only Alice can broadcast HTRD1a 1000
blocks after HT1a is broadcast since only Bob gave his signature for HTRD1a to
Alice. This transaction can be revocable when another transaction supersedes
HTRD1a using $multisig(P_{Alice4}, P_{Bob4})$ which does not have any block
maturity requirements.

\subsubsection{HTLC when the Receiver Broadcasts the Commitment Transaction}

For the potential receiver (Bob), the ``Timeout'' of receipt is refunded as an
HTLC Timeout Delivery transaction (HTD1b). This transaction directly refunds the
funds to the original sender (Alice) and is not encumbered in an RSMC. It
assumes that this HTLC has never been terminated off-chain, as Bob is attesting
that the broadcasted Commitment Transaction (C2b) is the most recent. If 3 days
have elapsed, Alice can broadcast HTD1b and take the refund. This transaction
consumes $multisig(P_{Alice5}, P_{Alice5})$ if Bob broadcasts C2b. Only Alice
can broadcast HTD1b since Bob gave his signature for HTD1b to Alice.

However, if HTD1b is not broadcast (3 days have not elapsed) and Bob knows the
preimage $R$, then Bob will be able to broadcast the HTLC Execution transaction
(HE1b) if he can produce $R$. This transaction is an RSMC. It consumes the
output $multisig(P_{Alice6}, P_{Bob6})$ and requires disclosure of $R$ if Bob
broadcasts C2b. The output for this transaction is an RSMC with
$multisig(P_{Alice7}, P_{Bob7})$ with relative maturity of 1000 blocks, and
$multisig(P_{Alice8}, P_{Bob8})$ which does not have any block maturity
requirements. Only Bob can broadcast HE1b since only Alice gave her signature
for HE1b to Bob.

After HE1b enters into the blockchain and 1000 block confirmations, a HTLC
Execution Revocable Delivery transaction (HERD1b) may be broadcast by Bob which
consumes $multisig(P_{Alice7}, P_{Bob7})$. Only Bob can broadcast HERD1b 1000
blocks after HE1b is broadcast since only Alice gave her signature for HERD1b to
Bob. This transaction can be revocable when another transaction supersedes
HERD1b using $multisig(P_{Alice8}, P_{Bob8})$ which does not have any block
maturity requirements.

\subsection{HTLC Off-chain Termination}

After an HTLC is constructed, to terminate an HTLC off-chain requires both
parties to agree on the state of the channel. If the recipient can prove
knowledge of $R$ to the counterparty, the recipient is proving that they are
able to immediately close out the channel on the Bitcoin blockchain and receive
the funds. At this point, if both parties wish to keep the channel open, they
should terminate the HTLC off-chain and create a new Commitment Transaction
reflecting the new balance.

\begin{figure}[H]
	%\vspace*{-2cm}
	\makebox[\linewidth]{
		\includegraphics[width=1.2\linewidth]{figures/htlc-newcommit.pdf}
	}
	\caption{Since Bob proved to Alice he knows $R$ by telling Alice $R$,
		Alice is willing to update the balance with a new Commitment
		Transaction. The payout will be the same whether C2 or C3 is
		broadcast at this time.
	}
\end{figure}

Similarly, if the recipient is not able to prove knowledge of $R$ by disclosing
$R$, both parties should agree to terminate the HTLC and create a new Commitment
Transaction with the balance in the HTLC refunded to the sender.

If the counterparties cannot come to an agreement or become otherwise
unresponsive, they should close out the channel by broadcasting the necessary
channel transactions on the Bitcoin blockchain.

However, if they are cooperative, they can do so by first generating a new
Commitment Transaction with the new balances, then invalidate the prior
Commitment by exchanging Breach Remedy transactions (BR2a/BR2b). Additionally,
if they are terminating a particular HTLC, they should also exchange some of
their own private keys used in the HTLC transactions. 

For example, Alice wishes to terminate the HTLC, Alice will disclose
$K_{Alice1}$ and $K_{Alice4}$ to Bob. Correspondingly if Bob wishes to terminate
the HTLC, Bob will disclose $K_{Bob1}$ and $K_{Bob4}$ to Alice. After the
private keys are disclosed to the counterparty, if Alice broadcasts C2a, Bob
will be able to take all the funds from the HTLC immediately. If Bob broadcasts
C2b, Alice will be able to take all funds from the HTLC immediately. Note that
when an HTLC is terminated, the older Commitment Transaction must be revoked as
well.

\begin{figure}[H]
	\vspace*{-3cm}
	\makebox[\linewidth]{
		\includegraphics[width=1.3\linewidth]{figures/Diagram2.pdf}
	}
	\caption{A fully revoked Commitment Transaction and terminated
		HTLC. If either party broadcasts Commitment 2, they will lose
		all their money to the counterparty. Other commitments (e.g. if
		Commitment 3 is the current Commitment) are not displayed for
		brevity.
	}
\end{figure}

Since both parties are able to prove to each other about the current state to
each other, they can come to agreement on the current balance inside the
channel. Since they may broadcast the current state on the blockchain, they are
able to come to agreement on netting out and terminating the HTLC with a new
Commitment Transaction.

\subsection{HTLC Formation and Closing Order}

To create a new HTLC, it is the same process as creating a new Commitment
Transaction, except the signatures for the HTLC are exchanged before the new
Commitment Transaction's signatures.

To close out an HTLC, the process is as follows (from C2 to C3):

\begin{enumerate}
	\item Alice signs and sends her signature for RD3b and C3b. At this
		point Bob can elect to broadcast C3b or C2b (with the HTLC) with
		the same payout. Bob is willing after receiving C3b to close out
		C2b.
	\item Bob signs and sends his signature for RD3a and C3a, as well as his
		private keys used for Commitment 2 and the HTLC being
		terminated; he sends Alice $K_{BobRSMC2}$, $K_{Bob5}$, and
		$K_{Bob8}$. At this point Bob should only broadcast C3b and
		should not broadcast C2b as he will lose all his money if he
		does so. Bob has fully revoked C2b and the HTLC. Alice is
		willing after receiving C3a to close out C2b.
	\item Alice signs and sends her signature for RD3b and C3b, as well as
		her private keys used for Commitment 2 and the HTLC being
		terminated; she sends Bob $K_{AliceRSMC2}$, $K_{Bob1}$, and
		$K_{Bob4}$. At this point neither party should broadcast
		Commitment 2, if they do so, their funds will be going to the
		counterparty. The old Commitment and old HTLC are now revoked
		and fully terminated. Only the new Commitment 3 remains, which
		does not have at HTLC.
\end{enumerate}

When the HTLC has been closed, the funds are updated so that the present balance
in the channel is what would occur had the HTLC contract been completed and
broadcast on the blockchain. Instead, both parties elect to do off-chain
novation and update their payments inside the channel.

It is absolutely necessary for both parties to complete off-chain novation
within their designated time window. For the receiver (Bob), he must know $R$
and update his balance with Alice within 3 days (or whatever time was selected),
else Alice will be able to redeem it within 3 days. For Alice, very soon after
her timeout becomes valid, she must novate or broadcast the HTLC Timeout
transaction. She must also novate or broadcast the HTLC Timeout Revocable
Delivery transaction as soon as it becomes valid. If the counterparty is
unwilling to novate or is stalling, then one must broadcast the current channel
state, including HTLC transactions) onto the Bitcoin blockchain.

The amount of time flexibility with these offers to novate are dependent upon
one's contingent dependencies on the hashlock $R$. If one establishes a contract
that the HTLC must be resolved within 1 day, then if the transaction times out
Alice must resolve it by day 4 (3 days plus 1), else Alice risks losing funds.

\section{Key Storage}

Keys are generated using BIP 0032 Hierarchical Deterministic
Wallets\cite{bip32}. Keys are pre-generated by both parties. Keys are generated
in a merkle tree and are very deep within the tree. For instance, Alice
pre-generates one million keys, each key being a child of the previous key.
Alice allocates which keys to use according to some deterministic manner. For
example, she starts with the child deepest in the tree to generate many sub-keys
for day 1. This key is used as a master key for all keys generated on day 1. She
gives Bob the address she wishes to use for the next transaction, and discloses
the private key to Bob when it becomes invalidated. When Alice discloses to Bob
all private keys derived from the day 1 master key and does not wish to continue
using that master key, she can disclose the day 1 master key to Bob. At this
point, Bob does not need to store all the keys derived from the day 1 master
key. Bob does the same for Alice and gives her his day 1 key.

When all Day 2 private keys have been exchanged, for example by day 5, Alice
discloses her Day 2 key. Bob is able to generate the Day 1 key from the Day 2
key, as the Day 1 key is a child of the Day 2 key as well.

If a counterparty broadcasts the wrong Commitment Transaction, which private
key to use in a transaction to recover funds can either be brute forced, or if
both parties agree, they can use the sequence id number when creating the
transaction to identify which sets of keys are used.

This enables participants in a channel to have prior output states
(transactions) invalidated by both parties without using much data at all. By
disclosing private keys pre-arranged in a merkle-tree, it is possible to
invalidate millions of old transactions with only a few kilobytes of data per
channel. Core channels in the Lightning Network can conduct billions of
transactions without a need for significant storage costs.

\section{Blockchain Transaction Fees for Bidirectional Channels}

It is possible for each participant to generate different versions of
transactions to ascribe blame as to who broadcast the transaction on the
blockchain. By having the knowledge who broadcast a transaction and the ability
to ascribe blame, a third party service can be used to hold fees in a 2-of-3
multisig escrow. If one wishes to broadcast the transaction chain instead of
agreeing to do a Funding Close or replacement with a new Commitment Transaction,
one would communicate with the third party and broadcast it to the blockchain.
If the counterparty refuses the notice from the third party to cooperate, the
penalty is rewarded to the non-cooperative party. In most instances,
participants may be indifferent to the transaction fees in the event of an
uncooperative counterparty.

One should pick counterparties in the channel who will be cooperative, but is
not an absolute necessity for the system to function. Note that this does not
require trust among the rest of the network, and is only relevant for the
comparatively minor transaction fees. The less trusted party may just be the one
responsible for transaction fees.

The Lightning Network fees will likely be significantly lower than blockchain
transaction fees. The fees are largely derived from the time-value of locking up
funds for a particular route, as well as paying for the chance of channel close
on the blockchain. These should be significantly lower than on-chain
transactions, as many transactions on a Lightning Network channel can be settled
into one single blockchain transaction. With a sufficiently robust and
interconnected network, the fees should asymptotically approach negligibility
for many types of transactions. With cheap fees and fast transactions, it will
be possible to build scalable micropayments, even amongst high-frequency systems
such as Internet of Things applications or per-unit micro-billing.

\section{Pay to Contract}

It is possible construct a cryptographically provable ``Delivery Versus
Payment'' contract, or pay-to-contract\cite{paytocontract}, as proof of
payment. This proof can be established as knowledge of the input R from hash(R)
as payment of a certain value. By embedding the contract between the buyer and
seller that knowing R is proof of funds sent, the recipient of funds has no
incentive to disclose R unless they have certainty that they will receive
payment. When the funds eventually get pulled from the buyer by their
counterparty in their micropayment channel, R is disclosed as part of that pull
of funds. One can design paper legal documents that specify that knowledge or
disclosure of R implies fulfillment of payment. The sender can arrange a
cryptographically signed contract for knowledge of inputs for hashes as
fulfillment of contracts before payment.

\section{The Bitcoin Lightning Network}

By having a micropayment channel with contracts encumbered by hashlocks and
timelocks, it is possible to clear transactions over a multi-hop payment
network using a series of decrementing timelocks without additional central
clearinghouses.

Traditionally, financial markets clear transactions by transferring the
obligation for delivery at a central point and settle by transferring ownership
through this central hub. Bank wire and fund transfer systems (such as ACH and
the Visa card network), or equities clearinghouses (such as the DTCC) operate
in this manner.

As Bitcoin enables programmatic money, it is possible to create transactions
without contacting a central clearinghouse. Transactions can execute off-chain
with no third party which collects all funds before disbursing it -- only
transactions with uncooperative channel counterparties become automatically
adjudicated on the blockchain. 

The obligation to deliver funds to an end-recipient is achieved through a
process of chained delegation. Each participant along the path assumes the
obligation to deliver to a particular recipient. They pass on this obligation
to the next participant in the path. The obligation of each subsequent
participant along the path, defined in their respective HTLCs, has a shorter
time to completion compared to the prior participant. This way each participant
is sure that they will be able to claim funds when the obligation is sent along
the path.

Bitcoin Transaction Scripting, a form of what some call an implementation of
``Smart Contracts''\cite{smartcontracts}, enables systems without trusted
custodial clearinghouses or escrow services.

\subsection{Decrementing Timelocks}

Presume Alice wishes to send 0.001 BTC to Dave. She locates a route through Bob
and Carol. The transfer path would be Alice to Bob to Carol to Dave.

\begin{figure}[H]
	\makebox[\linewidth]{
		\includegraphics[width=1\linewidth]{figures/network-fig0.pdf}
	}
	\caption{Payment over the Lightning Network using HTLCs.
	}
\end{figure}

When Alice sends payment to Dave through Bob and Carol, she requests from Dave
hash(R) to use for this payment. Alice then counts the amount of hops until the
recipient and uses that as the HTLC expiry. In this case, she sets the HTLC
expiry at 3 days. Bob then creates an HTLC with Carol with an expiry of 2 days,
and Carol does the same with Dave with an expiry of 1 day. Dave is now free to
disclose R to Carol, and both parties will likely agree to immediate settlement
via novation with a replacement Commitment Transaction. This then occurs
step-by-step back to Alice. Note that this occurs off-chain, and nothing is
broadcast to the blockchain when all parties are cooperative.

\begin{figure}[H]
	\makebox[\linewidth]{
		\includegraphics[width=1\linewidth]{figures/network-fig1.pdf}
	}
	\caption{Settlement of HTLC, Alice's funds get sent to Dave.
	}
\end{figure}

Decrementing timelocks are used so that all parties along the path know that
the disclosure of R will be able to pull funds, since they will at worst be
pulling funds after the date whereby they must receive R. If Dave does not
produce R within 1 day to Carol, then Carol will be able to close out the HTLC.
If Dave broadcasts R after 1 day, then he will not be able to pull funds from
Carol. Carol's responsibility to Bob occurs on day 2, so Carol will never be
responsible for payment to Dave without an ability to pull funds from Bob
provided that she updates her transaction with Dave to the blockchain or via
novation.

In the event that R gets disclosed to the participants halfway through expiry
along the path (e.g. day 2), then it is possible for some parties along the
path to be enriched. The sender will be able to know R, so due to Pay to
Contract, the payment will have been fulfilled even though the receiver did not
receive the funds. Therefore, the receiver will never disclose R unless they
have received an HTLC from their channel counterparty; they are guaranteed to
receive payment from one of their channel counterparties upon disclosure of the
preimage.

In the event a party outright disconnects, the counterparty will be responsible
for broadcasting the current Commitment Transaction state in the channel to the
blockchain. Only the failed non-responsive channel state gets closed out on the
blockchain, all other channels should continue to update their Commitment
Transactions via novation inside the channel. Therefore, counterparty risk for
transaction fees are only exposed to direct channel counterparties. If a node
along the path decides to become unresponsive, the participants not directly
connected to that node suffer only decreased time-value of their funds by not
conducting early settlement before the HTLC close.

\begin{figure}[H]
	\makebox[\linewidth]{
		\includegraphics[width=1\linewidth]{figures/network-fig2.pdf}
	}
	\caption{Only the non-responsive channels get broadcast on the
		blockchain, all others are settled off-chain via novation.
	}
\end{figure}

\subsection{Payment Amount}

It is necessary to use a small payment per HTLC. One should not use an extremely
high payment, in case the payment does not fully route to its destination. If
the payment does not reach its destination and one of the participants along the
path is uncooperative, it is possible that the sender must wait until the expiry
before receiving a refund. Delivery may be lossy, similar to packets on the
internet, but the network cannot outright steal funds in transit. Since
transactions don't hit the blockchain with cooperative channel counterparties,
it is recommended to use as small of a payment as possible. A tradeoff exists
between locking up transaction fees on each hop versus the desire to use as
small transaction amounts as possible (which the latter may incur higher total
fees). Smaller transfers with more intermediaries imply a higher percentage paid
as Lightning Network fees to the intermediaries.

\subsection{Clearing Failure and Rerouting}

If a transaction fails to reach its final destination, the receiver should send
an equal payment to the sender with the same hash, but not disclose R. This
will net out the disclosure of the hash for the sender, but may not for the
receiver. The receiver, who generated the hash, should discard R and never
broadcast it. If one channel along the path cannot be contacted, then the
channels may elect to wait until the path expires, which all participants will
likely close out the HTLC as unsettled without any payment with a new
Commitment Transaction.

\begin{figure}[H]
	\makebox[\linewidth]{
		\includegraphics[width=1\linewidth]{figures/network-fig3.pdf}
	}
	\caption{Dave creates a path back to Alice after Alice fails to send
		funds to Dave, because Carol is uncooperative. The input R from
		hash(R) is never brodcast by Dave, because Carol did not
		complete her actions. If R was broadcast, Alice will
		break-even. Dave, who controls R should never broadcast R
		because he may not receive funds from Carol, he should let the
		contracts expire. Alice and Bob have the option to net out and
		close the contract early, as well, in this diagram.
	}
\end{figure}

If the refund route is the same as the payment route, and there are no
half-signed contracts whereby one party may be able to steal funds, it is
possible to outright cancel the transaction by replacing it with a new
Commitment Transaction starting with the most recent node who participated in
the HTLC.

It is also possible to clear out a channel by creating an alternate route path
in which payment will occur in the opposite direction (netting out to zero)
and/or creating an entirely alternate route for the payment path. This will
create a time-value of money for disclosing inputs to hashes on the Lightning
Network. Participants may specialize in high connectivity between nodes and
offering to offload contract hashlocks from other nodes for a fee. These
participants will agree to payments which net out to zero (plus fees), but are
loaning bitcoins for a set time period. Most likely, these entities with low
demand for channel resources will be end-users who are already connected to
multiple well-connected nodes. When an end-user connects to a node, the node
may ask the client to lock up their funds for several days to another channel
the client has established for a fee. This can be achieved by having the new
transactions require a new hash(Y) from input Y in addition to the existing
hash which may be generated by any participant, but must disclose Y only after
a full circle is established. The new participant has the same responsibility
as well as the same timelocks as the old participant being replaced. It is also
possible that the one new participant replaces multiple hops.

\begin{figure}[H]
	\makebox[\linewidth]{
		\includegraphics[width=1\linewidth]{figures/network-fig4.pdf}
	}
	\caption{Erin is connected to both Bob and Dave. If Bob wishes to free
		up his channel with Carol, since that channel is active and
		very profitable, Bob can offload the payment to Dave via Erin.
		Since Erin has extra bitcoin available, she will be able to
		collect some fee for offloading the channel between Bob and
		Carol as well as between Carol and Dave. The channels between
		Bob and Carol as well as Carol and Dave are undone and no
		longer have the HTLC, nor has payment occurred on that path, it
		will occur on the path involving Erin. This is achieved by
		creating a new payment from Dave to Carol to Bob contingent
		upon Erin constructing an HTLC. The payment in dashed lines
		(red) are netted out to zero and settled via a new Commitment
		Contract.
	}
\end{figure}

\subsection{Payment Routing}

It is theoretically possible to build a route map implicitly from observing
2-of-2 multisigs on the blockchain to build a routing table. Note, however, this
is not feasible with pay-to-script-hash transaction outputs, which can be
resolved out-of-band from the bitcoin protocol via a third party routing
service. Building a routing table will become necessary for large operators
(e.g. BGP, CJBDNS). Eventually, with optimizations, the network will look a lot
like the correspondent banking network, or Tier-1 ISPs. Similar to how packets
still reach their destination on your home network connection, not all
participants need to have a full routing table. The core Tier-1 routes can be
online all the time \textemdash while nodes at the edges, such as average users,
would be connected intermittently.

Node discovery can occur along the edges by pre-selecting and offering partial
routes to well-known nodes.

\subsection{Fees}

Lightning Network Fees, which differ from blockchain fees, are paid directly
between participants within the channel. The fees pay for the time-value of
money for consuming the channel for a determined maximum period of time, and
for counterparty risk of non-communication.

Counterparty risk for fees only exist with one's direct channel counterparty.
If a node two hops away decides to disconnect and their transaction gets
broadcast on the blockchain, one's direct counterparties should not broadcast
on the blockchain, but continue to update via novation with a new Commitment
Transaction. See the HTLC section on fees for more information about
counterparty risk.

The time-value of fees pays for consuming time (e.g. 3 days) and is
conceptually equivalent to a gold lease rate without custodial risk; it is the
time-value for using up the access to money for a very short duration. Since
certain paths may become very profitable in one direction, it is possible for
fees to be negative to encourage the channel to be available for those
profitable paths.

\section{Risks}

The primary risks relate to timelock expiration. Additionally, for core nodes
and possibly some merchants to be able to route funds, the keys must be held
online for lower latency. However, end-users and nodes are able to keep their 
their private keys firewalled off in cold storage.

\subsection{Improper Timelocks}

Participants must choose timelocks with sufficient amount of time. If
insufficient time is given, it is possible that transactions believed to be
invalid will become valid, enabling coin theft by the counterparty. There is a
trade-off between longer timelocks and the time value of money. When writing
wallet and Lightning Network application software, it is necessary to ensure
that sufficient time is given and users are able to have their transactions
enter into the blockchain when interacting with non-cooperative or malicious
channel counterparties.

\subsection{Forced Expiration Spam}

Forced expiration of many transactions may be the greatest systemic risk when
using the Lightning Network. If a malicious participant creates many channels
and forces them all to expire at once, these may overwhelm block data capacity,
forcing expiration and broadcast to the blockchain. The result would be mass
spam on the bitcoin network. The spam may delay transactions to the point where
other locktimed transactions become valid.

This may be mitigated by permitting one transaction replacement on all pending
transactions. Anti-spam can be used by permitting only one transaction
replacement of a higher sequence number by the inverse of an even or odd number.
For example, if an odd sequence number was broadcasted, permit a replacement to
a higher even number only once. Transactions would use the sequence number in an
orderly way to replace other transactions. This mitigates the risk assuming
honest miners. This attack is extremely high risk, as incorrect broadcast of
Commitment Transactions entail a full penalty of all funds in the channel.

Additionally, one may attempt to steal HTLC transactions by forcing a timeout
transaction to go through when it should not. This can be easily mitigated by
having each transfer inside the channel be lower than the total transaction
fees used. Since transactions are extremely cheap and do not hit the blockchain
with cooperative channel counterparties, large transfers of value can be split
into many small transfers. This attempt can only work if the blocks are
completely full for a long time. While it is possible to mitigate it using a
longer HTLC timeout duration, variable block sizes may become common, which may
need mitigations.

If this type of transaction becomes the dominant form of transactions which are
included on the blockchain, it may become necessary to increase the block size
and run a variable blocksize structure and timestop flags as described in the
section below. This can create sufficient penalties and disincentives to be
highly unprofitable and unsuccessful for attackers, as attackers lose all their
funds from broadcasting the wrong transaction, to the point where it will never
occur.

\subsection{Coin Theft via Cracking}

As parties must be online and using private keys to sign, there is a
possibility that if one's computer is compromised, that coins will be stolen by
the counterparty. While there may be methods to mitigate the threat for the
sender and the receiver, the intermediary nodes must be online and will likely
be processing the transaction automatically. For this reason, the intermediary
nodes will be at risk and should not be holding a substantial amount of money
in this ``hot wallet.'' Intermediary nodes which have better security will
likely be able to out-compete others in the long run and be able to conduct
greater transaction volume due to lower fees. Historically, one of the largest
component of fees and interest in the financial system are from various forms of
counterparty risk -- in Bitcoin it is possible that the largest component in
fees will be derived from security risk premiums.

A Funding Transaction may have multiple outputs with multiple Commitment
Transactions, with the Funding Transaction and some Commitment Transactions
stored offline. It is possible to create an equivalent of a ``Checking
Account'' and ``Savings Account'' by moving funds between outputs from a
Funding Transaction, with the ``Savings Account'' stored offline and requiring
additional signatures from security services.

\subsection{Data Loss}

When one party loses data, it is possible for the counterparty to steal funds.
This can be mitigated by having a third party data storage service where
encrypted data gets sent to this third party service which the party cannot
decrypt. Additionally, one should choose channel counterparties who are
responsible and willing to provide the current state, with some periodic tests
of honesty.

\subsection{Forgetting to Broadcast the Transaction in Time}

If one does not broadcast a transaction at the correct time, the counterparty
may steal funds. This can be mitigated by having a designated third party to send
funds. An output fee can be added to create an incentive for this third party to
watch the network. Further, this can also be mitigated by implementing relative
OP\_CHECKLOCKTIMEVERIFY.

\subsection{Inability to Make Necessary Soft-Forks}

Changes are necessary to bitcoin, such as the malleability soft-fork.
Additionally, if this system becomes popular, it will be necessary for the
system to securely transact with many users and some kind of structure like a
blockheigh timestop will be desirable. This system assumes such changes to
enable Lightning Network to exist entirely, as well as soft-forks ensuring the
security is robust against attackers will occur. While the system may continue
to operate with only some time lock and malleability soft-forks, there will be
necessary soft-forks regarding systemic risks. Without proper community
foresight, an inability to establish a timestop or similar function will allow
systemic attacks to take place and may not be recognized as imperative until an
attack actually occurs.

\subsection{Colluding Miner Attacks}

Miners may elect to refuse to enter in particular transactions (e.g. Breach
Remedy transactions) in order to assist in timeout coin theft. An attacker can
pay off all miners to refuse to include certain transactions in their mempool
and blocks. The miners can identify their own blocks in an attempt to prove
their behavior to the paying attacker.

This can be mitigated by encouraging miners to avoid identifying their own
blocks. Further, it should be expected that this kind of payment to miners is
malicious activity and the contract is unenforcible. Miners may then take
payment and surreptitiously mine a block without identifying the block to the
attacker. Since the attacker is paying for this, they will quickly run out of
money by losing the fee to the miner, as well as losing all their money in the
channel. This attack is unlikely and fairly unattractive as it is far too
difficult and requires a high degree of collusion with extreme risk.

The risk model of this occuring is similar to that of miners colluding to do reorg
attacks, exremely unlikely with many uncoordinated miners.

\section{Block Size Increases and Consensus}

If we presume that a decentralized payment network exists and one person will
make 3 blockchain transactions per year on average, Bitcoin will be able to
support over 35 million users with 1MB blocks in ideal circumstances (assuming
2000 transactions per MB). This is quite limited, and an increase of the block
size may be necessary to support everyone in the world using Bitcoin. A simple
increase of the block size would be a hard fork, meaning all nodes will need to
update their wallets if they wish to participate in the network with the larger
blocks.

While it may appear as though this system will mitigate the block size increases
in the short term, if it achieves global scale, it will necessitate a block size
increase in the long term. Creating a credible threat that spamming the
blockchain to encourage transactions to timeout becomes imperative.

To mitigate timelock spam vulnerabilities, non-miner and miners' consensus rules
may also differ if the miners' consensus rules are more restrictive. Non-miners
may accept blocks over 1MB, while miners may have different soft-caps on block
sizes. If a block size is above that cap, then that is viewed as an invalid
block by other miners, but not by non-miners. The miners will only build the
chain on blocks which are valid according to the agreed-upon soft-cap. This
permits miners to agree on raising the block size limit without requiring
frequent hard-forks from clients, so long as the amount raised by miners does
not go over the clients' hard limit. This mitigates the risk of mass-expiry of
transactions at once. All transactions which are not redeemed via Exercise
Settlement (ES) may have a very high fee attached, and miners may use a
consensus rule whereby those transactions are exempted from the soft-cap, making
it very likely the correct transactions will enter the blockchain.

When transactions are viewed as circuits and contracts instead of transaction
packets, the consensus risks can be measured by the amount of time available to
cover the UTXO set controlled by hostile parties. In effect, the upper bound of
the UTXO size is determined by transaction fees and the standard minimum
transaction output value. If the bitcoin miners have a deterministic mempool
which prioritizes transactions respecting a ``weak'' local time order of
transactions, it could become extremely unprofitable and unlikely for an attack
to succeed. Any transaction spam time attack by broadcasting the incorrect
Commitment Transaction is extremely high risk for the attacker, as it requires
an immense amount of bitcoin and all funds committed in those transactions will
be lost if the attacker fails.

\section{Use Cases}

In addition to helping bitcoin scale, there are many uses for transactions on
the Lightning Network:
\begin{itemize}
	\item Instant Transactions. Using Lightning, Bitcoin transactions are now nearly instant
		with any party. It is possible to pay for a cup of coffee with
		direct non-revocable payment in milliseconds to seconds.
	
	\item Exchange Arbitrage. There is presently incentive to hold funds on
		exchanges to be ready for large market moves due to 3-6 block
		confirmation times. It is possible for the exchange to
		participate in this network and for clients to move their funds
		on and off the exchange for orders nearly instantly. If the
		exchange does not have deep market depth and commits to only
		permitting limit orders close to the top of the order book, then
		the risk of coin theft becomes much lower. The exchange, in
		effect, would no longer have any need for a cold storage wallet.
		This may substantially reduce thefts and the need for trusted
		third party custodians.

	\item Micropayments. Bitcoin blockchain fees are far too high to accept
		micropayments, especially with the smallest of values. With this
		system, near-instant micropayments using Bitcoin without a 3rd
		party custodian would be possible. It would enable, for example,
		paying per-megabyte for internet service or per-article to read
		a newspaper.
	
	\item Financial Smart Contracts and Escrow. Financial contracts are
		especially time-sensitive and have higher demands on blockchain
		computation. By moving the overwhelming majority of trustless
		transactions off-chain, it is possible to have highly complex
		transaction contract terms without ever hitting the blockchain.

\end{itemize}

\section{Conclusion}

Creating a network of micropayment channels enables bitcoin scalability,
micropayments down to the satoshi, and near-instant transactions. These channels
represent real Bitcoin transactions, using the Bitcoin scripting opcodes to
enable the transfer of funds without risk of counterparty theft, especially with
long-term miner risk mitigations.

If all transactions using Bitcoin were on the blockchain, to enable 7 billion
people to make two transactions per day, it would require 24GB blocks every ten
minutes at best (presuming 250 bytes per transaction and 144 blocks per day).
Conducting all global payment transactions on the blockchain today implies
miners will need to do an incredible amount of computation, severely limiting
bitcoin scalability and full nodes to a few centralized processors.

If all transaction using Bitcoin were conducted inside a network of micropayment
channels, to enable 7 billion people to make two channels per year with
unlimited transactions inside the channel, it would require 133 MB blocks
(presuming 500 bytes per transaction and 52560 blocks per year). Current
generation desktop computers will be able to run a full node with old blocks
pruned out on 2TB of storage.

With a network of instantly confirmed micropayment channels whose payments are
encumbered by timelocks and hashlock outputs, Bitcoin can scale to billions of
users without custodial risk or blockchain centralization when transactions are
conducted securely off-chain using bitcoin scripting, with enforcement of
non-cooperation by broadcasting signed multisignature transactions on the
blockchain.

\section{Acknowledgements}

Micropayment channels have been developed by many parties, and have been
discussed on bitcointalk, the bitcoin mailing list, and IRC. The amount of
contributors to this idea are immense and much thought have been put into this
ability. Effort has been placed into citing and finding similar ideas, however
it is absolutely not near complete. In particular, there are many similarities
to a proposal by Alex Akselrod by using hashlocking as a method of encumbering a
hub-and-spoke payment channel.

Thanks to Peter Todd for correcting a significant error in the HTLC script, as
well as optimizing the opcode size.

Thanks to Elizabeth Stark for reviewing and corrections.

Thanks to Rusty Russell for reviewing this document and suggestions for making
the concept more digestible, as well as working on a construction which may
provide a stop-gap solution before a long-term malleability fix (to be described
in a future version).

\begin{appendices}
\section{Resolving Malleability}

In order to create these contracts in Bitcoin without a third party trusted
service, Bitcoin must fix the transaction malleability problem. If transactions
can be mutated, then signatures can be invalidated, thereby making refund
transactions and commitment bonds invalidated. This creates an opportunity for
hostile actors to use it as an opportunity for a negotiating tactic to steal
coins, in effect, a hostage scenario.

To mitigate malleability, it is necessary to make a soft-fork change to bitcoin.
Older clients would still work, but miners would need to update. Bitcoin has had
several soft forks in the past, including pay-to-script-hash (P2SH).

To mitigate malleability, it requires changing which contents are signed by the
participants. This is achieved by creating new Sighash types. In order to
accommodate this new behavior, a new P2SH type or new OP\_CHECKSIG is necessary
to make it a soft-fork rather than a hard-fork. 

Either may be used, if a new P2SH was defined, it would use a different output
script such as:
\\\\
\noindent\texttt{OP\_DUP OP\_HASH160 \textless 20-byte hash\textgreater
~OP\_EQUALVERIFY}
\\\\
Since this will always resolve to true provided a valid redeemScript, all
existing clients will return true. 

This opens the scripting system to construct new rules, including new signature
validation rules. At least one new sighash would need to exist.

A SIGHASH\_NOINPUT, will not sign any input transactions IDs nor sign the
index. 

By using SIGHASH\_NOINPUT, it allows one to be assured that one's counterparty
cannot invalidate entire trees of chained transactions of potential contract
states which were previously agreed upon, using transaction ID mutation. With
the new sighash flags, it is possible to spend from a parent transaction even
though the transaction ID has changed, so long as the script evaluates as true
(i.e. a valid signature).

SIGHASH\_NOINPUT implies significant risk with address reuse, as it can work
with any transaction in which the sigScript returns as valid, so multiple
transactions with the same outputs are redeemable (provided the output values
are less).

Further, and just as importantly, it permits participants to sign spends of
transactions without knowing the signatures of the transaction being spent. By
solving malleability in the above manner, two parties may build contracts and
spending on transactions without either party having the ability to broadcast
that original transaction on the blockchain until both parties agree. With the
new sighash types, participants may build potential contract states and
potential payout conditions and agree upon all terms, before the contract may be
paid, broadcast, and executed upon without the need for a trusted third party.

Without SIGHASH\_NOINPUT, one cannot build outputs before the transaction can be
funded. It is as if one cannot make any agreements without committing funds
without knowing what one is committing to. SIGHASH\_NOINPUT allows one to build
redemption for transactions which do not yet exist; in other words one can form
agreements before funding the transaction if the output is a 2-of-2
multisignature transaction.

To use SIGHASH\_NOINPUT, one builds a Funding Transaction, and does not yet sign
it. This Funding Transaction does not need to use SIGHASH\_NOINPUT if it is
spending from a transaction which already entered into the blockchain. To spend
from a Funding Transaction with a 2-of-2 multisignature output which is not yet
signed and broadcast, however, requires using SIGHASH\_NOINPUT.

A further stop-gap solution using OP\_CHECKSEQUENCEVERIFY or a less-optimal use
of OP\_CHECKLOCKTIMEVERIFY will be described in a future paper by Rusty Foster.
An updated version of this paper will also include these constructions.

\end{appendices}

%bibliography
\bibliographystyle{unsrt}
\bibliography{new}

\end{document}

%TODO Remove "Fidelity Bond" ("fidelity bond") and replace with a name which
%refers to it as a contract instead.

